\documentclass[11pt,]{article}
\usepackage{lmodern}
\usepackage{amssymb,amsmath}
\usepackage{ifxetex,ifluatex}
\usepackage{fixltx2e} % provides \textsubscript
\ifnum 0\ifxetex 1\fi\ifluatex 1\fi=0 % if pdftex
  \usepackage[T1]{fontenc}
  \usepackage[utf8]{inputenc}
\else % if luatex or xelatex
  \ifxetex
    \usepackage{mathspec}
  \else
    \usepackage{fontspec}
  \fi
  \defaultfontfeatures{Ligatures=TeX,Scale=MatchLowercase}
\fi
% use upquote if available, for straight quotes in verbatim environments
\IfFileExists{upquote.sty}{\usepackage{upquote}}{}
% use microtype if available
\IfFileExists{microtype.sty}{%
\usepackage{microtype}
\UseMicrotypeSet[protrusion]{basicmath} % disable protrusion for tt fonts
}{}
\usepackage[margin=1in]{geometry}
\usepackage{hyperref}
\hypersetup{unicode=true,
            pdfborder={0 0 0},
            breaklinks=true}
\urlstyle{same}  % don't use monospace font for urls
\usepackage{graphicx,grffile}
\makeatletter
\def\maxwidth{\ifdim\Gin@nat@width>\linewidth\linewidth\else\Gin@nat@width\fi}
\def\maxheight{\ifdim\Gin@nat@height>\textheight\textheight\else\Gin@nat@height\fi}
\makeatother
% Scale images if necessary, so that they will not overflow the page
% margins by default, and it is still possible to overwrite the defaults
% using explicit options in \includegraphics[width, height, ...]{}
\setkeys{Gin}{width=\maxwidth,height=\maxheight,keepaspectratio}
\IfFileExists{parskip.sty}{%
\usepackage{parskip}
}{% else
\setlength{\parindent}{0pt}
\setlength{\parskip}{6pt plus 2pt minus 1pt}
}
\setlength{\emergencystretch}{3em}  % prevent overfull lines
\providecommand{\tightlist}{%
  \setlength{\itemsep}{0pt}\setlength{\parskip}{0pt}}
\setcounter{secnumdepth}{0}
% Redefines (sub)paragraphs to behave more like sections
\ifx\paragraph\undefined\else
\let\oldparagraph\paragraph
\renewcommand{\paragraph}[1]{\oldparagraph{#1}\mbox{}}
\fi
\ifx\subparagraph\undefined\else
\let\oldsubparagraph\subparagraph
\renewcommand{\subparagraph}[1]{\oldsubparagraph{#1}\mbox{}}
\fi

%%% Use protect on footnotes to avoid problems with footnotes in titles
\let\rmarkdownfootnote\footnote%
\def\footnote{\protect\rmarkdownfootnote}

%%% Change title format to be more compact
\usepackage{titling}

% Create subtitle command for use in maketitle
\newcommand{\subtitle}[1]{
  \posttitle{
    \begin{center}\large#1\end{center}
    }
}

\setlength{\droptitle}{-2em}
  \title{}
  \pretitle{\vspace{\droptitle}}
  \posttitle{}
  \author{}
  \preauthor{}\postauthor{}
  \date{}
  \predate{}\postdate{}

\usepackage{enumitem}

\begin{document}

\begin{flushright} 2018 \end{flushright}

\textbf{\Large Baloney Detection Kit} \newline
\textbf{Evaluating claims in the media} \newline
Easton White

\section{Introduction}

There are lots of sources of information and mis-information online and
in the media. How do we evaluate articles? How can we be sure the
information within them is the truth? This assignment is intended to
help you critically analyze a recent news article.

\section{Your assignment}

This assignment consists of two main parts. First, choose a recent news
article from a local or national US newspaper. Find an article on a
topic that interests you. The article should be a piece of news and not
an opinion piece.

Your job is to evaluate the claims of the article. Is it trustworthy?
Why?

Read the article, and the ask yourself the following questions:

\begin{itemize}
\tightlist
\item
  Who wrote the article? Do they have any bias?
\item
  Where was the article published? Is it a reliable source?
\item
  Are there any numbers or figures referenced in the article? If so,
  verify that these numbers are correct by finding them from their
  original data sources.
\item
  What important information is potentially missing from the article?
\end{itemize}

\section{How you will be graded}

\begin{table}[!h]
\begin{tabular}{l | l}
    Criteria & Points \\ \hline \hline
    Addressed questions in previous section & 10 \\
    Quality of writing & 5 \\
    Formatting and referencing the source & 5 \\ \hline
    Total & 20
\end{tabular}
\end{table}

Quality of writing depends on the logical argument presented, sentence
structure, ability to organize the information into paragraphs, and
grammar. Please be sure to re-read your own paper, have friends read it,
and even take it to a writing specialist in the Student Academic Success
Center. I expect well-crafted essays that have been edited.

Your article summary should be one page, double-spaced, and 12 point
font. You also need to include a reference to the article that you are
evaluating. Also, be sure to reference any articles or websites that you
use in evaluating the article.


\end{document}
