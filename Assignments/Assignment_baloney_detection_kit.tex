\documentclass[11pt,]{article}
\usepackage{lmodern}
\usepackage{amssymb,amsmath}
\usepackage{ifxetex,ifluatex}
\usepackage{fixltx2e} % provides \textsubscript
\ifnum 0\ifxetex 1\fi\ifluatex 1\fi=0 % if pdftex
  \usepackage[T1]{fontenc}
  \usepackage[utf8]{inputenc}
\else % if luatex or xelatex
  \ifxetex
    \usepackage{mathspec}
  \else
    \usepackage{fontspec}
  \fi
  \defaultfontfeatures{Ligatures=TeX,Scale=MatchLowercase}
\fi
% use upquote if available, for straight quotes in verbatim environments
\IfFileExists{upquote.sty}{\usepackage{upquote}}{}
% use microtype if available
\IfFileExists{microtype.sty}{%
\usepackage{microtype}
\UseMicrotypeSet[protrusion]{basicmath} % disable protrusion for tt fonts
}{}
\usepackage[margin=1in]{geometry}
\usepackage{hyperref}
\hypersetup{unicode=true,
            pdfborder={0 0 0},
            breaklinks=true}
\urlstyle{same}  % don't use monospace font for urls
\usepackage{graphicx,grffile}
\makeatletter
\def\maxwidth{\ifdim\Gin@nat@width>\linewidth\linewidth\else\Gin@nat@width\fi}
\def\maxheight{\ifdim\Gin@nat@height>\textheight\textheight\else\Gin@nat@height\fi}
\makeatother
% Scale images if necessary, so that they will not overflow the page
% margins by default, and it is still possible to overwrite the defaults
% using explicit options in \includegraphics[width, height, ...]{}
\setkeys{Gin}{width=\maxwidth,height=\maxheight,keepaspectratio}
\IfFileExists{parskip.sty}{%
\usepackage{parskip}
}{% else
\setlength{\parindent}{0pt}
\setlength{\parskip}{6pt plus 2pt minus 1pt}
}
\setlength{\emergencystretch}{3em}  % prevent overfull lines
\providecommand{\tightlist}{%
  \setlength{\itemsep}{0pt}\setlength{\parskip}{0pt}}
\setcounter{secnumdepth}{0}
% Redefines (sub)paragraphs to behave more like sections
\ifx\paragraph\undefined\else
\let\oldparagraph\paragraph
\renewcommand{\paragraph}[1]{\oldparagraph{#1}\mbox{}}
\fi
\ifx\subparagraph\undefined\else
\let\oldsubparagraph\subparagraph
\renewcommand{\subparagraph}[1]{\oldsubparagraph{#1}\mbox{}}
\fi

%%% Use protect on footnotes to avoid problems with footnotes in titles
\let\rmarkdownfootnote\footnote%
\def\footnote{\protect\rmarkdownfootnote}

%%% Change title format to be more compact
\usepackage{titling}

% Create subtitle command for use in maketitle
\newcommand{\subtitle}[1]{
  \posttitle{
    \begin{center}\large#1\end{center}
    }
}

\setlength{\droptitle}{-2em}
  \title{}
  \pretitle{\vspace{\droptitle}}
  \posttitle{}
  \author{}
  \preauthor{}\postauthor{}
  \date{}
  \predate{}\postdate{}

\usepackage{enumitem}

\begin{document}

\begin{flushright} 2018 \end{flushright}

\textbf{\Large Baloney Detection Kit} \newline
\textbf{Using the baloney detection kit in the 21st century} \newline
Easton White

\section{Introduction}

Sagan wrote his baloney detection kit over 20 years ago. There was fake
news and pseudoscience at that time as there is today. Your job in the
assignment will be to use the baloney detection kit to assess a claim in
the media. It might be useful to re-read the baloney detection kit from
the beginning of class.

\section{Your assignment}

This assignment consists of two main parts. First, choose a tool from
Sagan's baloney detection kit. Alternatively, you can create your own
tool. Explain how the tool works. Next, put the tool into action. Use it
to evaluate a claim in a newspaper article, by a politician, in an
advertisement, or another example.

In your writeup, be sure to address the following questions:

\begin{itemize}
\tightlist
\item
  How does the tool you have chosen work?
\item
  How does the tool help you understand the arguments made in the piece
  of media that you chose?
\end{itemize}

\section{How you will be graded}

\begin{table}[!h]
\begin{tabular}{l | l}
    Criteria & Points \\ \hline \hline
    Addressed questions in previous section & 10 \\
    Quality of writing & 5 \\
    Formatting and referencing the source & 5 \\ \hline
    Total & 20
\end{tabular}
\end{table}

Quality of writing depends on the logical argument presented, sentence
structure, ability to organize the information into paragraphs, and
grammar. Please be sure to re-read your own paper, have friends read it,
and even take it to a writing specialist in the Student Academic Success
Center. I expect well-crafted essays that have been edited.

Your article summary should be 1-2 pages, double-spaced, and 12 point
font. You also need to include a reference to the article that you are
evaluating. Also, be sure to reference any articles or websites that you
use in evaluating the article.


\end{document}
